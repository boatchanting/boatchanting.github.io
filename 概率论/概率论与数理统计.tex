\documentclass[UTF8]{ctexart}
\usepackage{tcolorbox}
\usepackage{mdframed}
\usepackage{amsmath}  
\usepackage{amssymb}  
\usepackage{graphicx}
\usepackage{float}  
\usepackage{wrapfig}  
\usepackage{algorithm}  
\usepackage{algorithmic} 
\usepackage{color,xcolor} 
\usepackage{colortbl}
\usepackage{graphicx}
\usepackage{booktabs} %绘制表格
\usepackage{caption2} %标题居中
\usepackage{geometry}
\usepackage{xcolor}
\usepackage{setspace}
\usepackage{geometry}
\geometry{a4paper, margin=1in}
\setstretch{1.2}
% 定义人物说话框的颜色
\newtcolorbox{dialoguebox}[2][]{colback=#1!5!white, colframe=#1!75!black, title=#2}
% 定义颜色
\definecolor{fomula}{RGB}{255, 230, 204} % 浅橙
\definecolor{problem}{RGB}{204, 255, 204} % 浅绿
\definecolor{definition}{RGB}{204, 229, 255} % 浅蓝
%\definecolor{}{RGB}{255, 204, 255} % 浅粉
%\definecolor{}{RGB}{220, 220, 220} % 浅灰
\usepackage{titlesec}
\titleformat{\section}
{\fontsize{20pt}{24pt}\selectfont\bfseries} % 20pt字体大小,24pt行距
{\thesection}{1em}{}
% 设置二级标题字体大小为18pt
\titleformat{\subsection}
{\fontsize{18pt}{22pt}\selectfont\bfseries} % 18pt字体大小,22pt行距
{\thesubsection}{1em}{}
%设定正文字体大小
\AtBeginDocument{\fontsize{15pt}{20pt}\selectfont}
\title{ \fontsize{24pt}{24pt}\selectfont 概率论与数理统计} %
\author{\fontsize{20pt}{20pt}\selectfont boatchanting} %
\date{}
\begin{document}
	\maketitle
	% 第一章:随机事件与概率
	
	\section{随机事件与概率}
	
	\subsection{随机事件及其运算}
	
	\definecolor{fomula}{RGB}{255, 230, 204} % 浅橙 
	\definecolor{definition}{RGB}{204, 229, 255} % 浅蓝
	\definecolor{problem}{RGB}{204, 255, 204} % 浅绿
	
	\begin{tcolorbox}[colback=definition!5!white, colframe=definition!75!black, title=事件的定义]
		在一次试验中,可能出现的结果称为基本事件,所有基本事件的集合称为样本空间,记作 $\Omega$。\\
		样本空间的子集称为事件。
	\end{tcolorbox}
	
	\begin{tcolorbox}[colback=definition!5!white, colframe=definition!75!black, title=随机事件]
		试验中每次可能发生或不发生的事件称为随机事件,用 $A, B, C$ 等表示。
	\end{tcolorbox}
	
	\begin{tcolorbox}[colback=definition!5!white, colframe=definition!75!black, title=事件的运算]
		\begin{itemize}
			\item 事件 $A$ 和 $B$ 的\textbf{并事件} $A \cup B$ 表示 $A$ 或 $B$ 发生。
			\item 事件 $A$ 和 $B$ 的\textbf{交事件} $A \cap B$ 表示 $A$ 和 $B$ 同时发生。
			\item 事件 $A$ 的\textbf{补事件} $\bar{A}$ 表示 $A$ 不发生。
		\end{itemize}
	\end{tcolorbox}
	
	\subsection{概率的公理化定义与概率的性质}
	
	\begin{tcolorbox}[colback=definition!5!white, colframe=definition!75!black, title=概率的公理化定义]
		设 $\Omega$ 为样本空间,其子集即为随机事件。概率是一个定义在事件上的函数 $P(\cdot)$,满足以下三条公理:
		\begin{enumerate}
			\item \textbf{非负性}:对任何事件 $A$,有 $P(A) \geq 0$。
			\item \textbf{规范性}:$P(\Omega) = 1$。
			\item \textbf{可加性}:若 $A$ 和 $B$ 为互不相容事件,则 $P(A \cup B) = P(A) + P(B)$。
		\end{enumerate}
	\end{tcolorbox}
	
	\begin{tcolorbox}[colback=fomula!5!white, colframe=fomula!75!black, title=概率的性质]
		对于任意两个事件 $A$ 和 $B$,概率 $P$ 满足以下性质:
		\begin{itemize}
			\item $P(\bar{A}) = 1 - P(A)$。
			\item $P(A \cup B) = P(A) + P(B) - P(A \cap B)$。
			\item 若 $A \subset B$,则 $P(A) \leq P(B)$。
		\end{itemize}
	\end{tcolorbox}
	
	\subsection{条件概率与事件的独立性}
	
	\begin{tcolorbox}[colback=definition!5!white, colframe=definition!75!black, title=条件概率]
		给定事件 $B$ 且 $P(B) > 0$,事件 $A$ 在 $B$ 发生条件下的\textbf{条件概率}定义为:
		\[
		P(A|B) = \frac{P(A \cap B)}{P(B)}
		\]
	\end{tcolorbox}
	
	\begin{tcolorbox}[colback=definition!5!white, colframe=definition!75!black, title=事件的独立性]
		如果两个事件 $A$ 和 $B$ 满足 $P(A \cap B) = P(A) \cdot P(B)$,则称 $A$ 和 $B$ 是\textbf{独立事件}。
	\end{tcolorbox}
	
	\subsection{全概率公式与贝叶斯公式}
	
	\begin{tcolorbox}[colback=definition!5!white, colframe=definition!75!black, title=全概率公式]
		设 $B_1, B_2, \dots, B_n$ 为一组两两互不相容且穷尽样本空间 $\Omega$ 的事件,且 $P(B_i) > 0$。对于任意事件 $A$,有:
		\[
		P(A) = \sum_{i=1}^n P(A | B_i) P(B_i)
		\]
	\end{tcolorbox}
	
	\begin{tcolorbox}[colback=definition!5!white, colframe=definition!75!black, title=贝叶斯公式]
		在全概率公式的条件下,对于事件 $A$ 和 $B_i$($i = 1, 2, \dots, n$),有:
		\[
		P(B_i | A) = \frac{P(A | B_i) P(B_i)}{\sum_{j=1}^n P(A | B_j) P(B_j)}
		\]
	\end{tcolorbox}
	% 第二章:离散型随机变量
	
	\section{离散型随机变量}
	
	\subsection{随机变量}
	
	\begin{tcolorbox}[colback=definition!5!white, colframe=definition!75!black, title=随机变量]
		设 $\Omega$ 为样本空间。若定义一个从样本空间到实数集合的映射 $X: \Omega \rightarrow \mathbb{R}$,则称 $X$ 为一个\textbf{随机变量}。即每个样本点 $\omega \in \Omega$ 通过 $X$ 映射为一个实数 $X(\omega)$。
	\end{tcolorbox}
	
	\subsection{一维离散型随机变量}
	
	\begin{tcolorbox}[colback=definition!5!white, colframe=definition!75!black, title=离散型随机变量]
		如果随机变量 $X$ 的取值是有限个或可列无限个,则称 $X$ 为\textbf{离散型随机变量}。
	\end{tcolorbox}
	
	\begin{tcolorbox}[colback=definition!5!white, colframe=definition!75!black, title=概率质量函数]
		设 $X$ 是一个离散型随机变量,且可能的取值为 $x_1, x_2, \dots$。定义
		\[
		p(x_i) = P(X = x_i), \quad i = 1, 2, \dots
		\]
		则函数 $p(x)$ 称为 $X$ 的\textbf{概率质量函数},满足
		\[
		\sum_{i} p(x_i) = 1, \quad p(x_i) \geq 0.
		\]
	\end{tcolorbox}
	
	\subsection{离散型随机变量的数学期望与方差}
	
	\begin{tcolorbox}[colback=definition!5!white, colframe=definition!75!black, title=数学期望]
		设 $X$ 是一个离散型随机变量,其概率质量函数为 $p(x)$。则 $X$ 的\textbf{数学期望} $E(X)$ 定义为:
		\[
		E(X) = \sum_{i} x_i p(x_i).
		\]
		数学期望可以理解为 $X$ 的平均值或重心。
	\end{tcolorbox}
	
	\begin{tcolorbox}[colback=definition!5!white, colframe=definition!75!black, title=方差]
		设 $X$ 的数学期望为 $E(X) = \mu$,则 $X$ 的\textbf{方差} $\text{Var}(X)$ 定义为:
		\[
		\text{Var}(X) = E\left[(X - \mu)^2\right] = \sum_{i} (x_i - \mu)^2 p(x_i).
		\]
		方差表示随机变量取值的离散程度。
	\end{tcolorbox}
	
	\subsection{常用离散型随机变量及其分布}
	
	\begin{tcolorbox}[colback=definition!5!white, colframe=definition!75!black, title=二项分布]
		若随机变量 $X$ 表示 $n$ 次独立重复试验中事件 $A$ 发生的次数,每次试验中事件 $A$ 发生的概率为 $p$,则 $X$ 服从\textbf{二项分布},记为 $X \sim B(n, p)$,其概率质量函数为:
		\[
		P(X = k) = \binom{n}{k} p^k (1 - p)^{n - k}, \quad k = 0, 1, \dots, n.
		\]
	\end{tcolorbox}
	
	\begin{tcolorbox}[colback=definition!5!white, colframe=definition!75!black, title=泊松分布]
		若随机变量 $X$ 表示在单位时间内发生事件的次数,且事件发生服从平均为 $\lambda$ 的泊松过程,则 $X$ 服从\textbf{泊松分布},记为 $X \sim \text{Poisson}(\lambda)$,其概率质量函数为:
		\[
		P(X = k) = \frac{\lambda^k e^{-\lambda}}{k!}, \quad k = 0, 1, 2, \dots.
		\]
	\end{tcolorbox}
	
	\begin{tcolorbox}[colback=definition!5!white, colframe=definition!75!black, title=几何分布]
		若随机变量 $X$ 表示在一次成功前需要进行的独立重复试验次数,单次试验成功的概率为 $p$,则 $X$ 服从\textbf{几何分布},其概率质量函数为:
		\[
		P(X = k) = (1 - p)^{k - 1} p, \quad k = 1, 2, \dots.
		\]
	\end{tcolorbox}
	
	\begin{tcolorbox}[colback=definition!5!white, colframe=definition!75!black, title=超几何分布]
		若随机变量 $X$ 表示从包含 $N$ 个个体的总体中抽取 $n$ 个个体,其中有 $M$ 个成功个体,则 $X$ 表示成功个体数,且 $X$ 服从\textbf{超几何分布},其概率质量函数为:
		\[
		P(X = k) = \frac{\binom{M}{k} \binom{N - M}{n - k}}{\binom{N}{n}}, \quad k = \max(0, n - (N - M)), \dots, \min(n, M).
		\]
	\end{tcolorbox}
	
	\subsection{离散型随机变量函数的分布律}
	
	\begin{tcolorbox}[colback=definition!5!white, colframe=definition!75!black, title=离散型随机变量函数的分布律]
		设 $X$ 是离散型随机变量,函数 $g(X)$ 定义了 $X$ 的一个新变量 $Y = g(X)$。若已知 $X$ 的概率质量函数 $p(x)$,则 $Y$ 的概率质量函数为:
		\[
		P(Y = y) = \sum_{x: g(x) = y} p(x).
		\]
		这种方法用于求离散型随机变量函数的分布。
	\end{tcolorbox}
	
	% 第三章:连续型随机变量
	
	\section{连续型随机变量}
	
	\subsection{随机变量的分布函数}
	
	\begin{tcolorbox}[colback=definition!5!white, colframe=definition!75!black, title=分布函数]
		设 $X$ 为随机变量,其分布函数(或称为累积分布函数)定义为:
		\[
		F_X(x) = P(X \leq x).
		\]
		分布函数 $F_X(x)$ 满足以下性质:
		\begin{enumerate}
			\item $0 \leq F_X(x) \leq 1$;
			\item $F_X(x)$ 为非减函数;
			\item $\lim_{x \to -\infty} F_X(x) = 0$ 且 $\lim_{x \to +\infty} F_X(x) = 1$。
		\end{enumerate}
	\end{tcolorbox}
	
	\subsection{连续型随机变量及其密度函数}
	
	\begin{tcolorbox}[colback=definition!5!white, colframe=definition!75!black, title=连续型随机变量]
		若随机变量 $X$ 的分布函数 $F_X(x)$ 是一个处处可导的连续函数,则称 $X$ 为\textbf{连续型随机变量}。
	\end{tcolorbox}
	
	\begin{tcolorbox}[colback=definition!5!white, colframe=definition!75!black, title=概率密度函数]
		对于连续型随机变量 $X$,存在一个非负函数 $f_X(x)$,使得对于任意 $a \leq b$,有
		\[
		P(a \leq X \leq b) = \int_{a}^{b} f_X(x) \, dx.
		\]
		该函数 $f_X(x)$ 称为 $X$ 的\textbf{概率密度函数},且满足
		\[
		\int_{-\infty}^{+\infty} f_X(x) \, dx = 1.
		\]
	\end{tcolorbox}
	
	\subsection{连续型随机变量的数学期望与方差}
	
	\begin{tcolorbox}[colback=definition!5!white, colframe=definition!75!black, title=数学期望]
		设 $X$ 是一个连续型随机变量,概率密度函数为 $f_X(x)$。则 $X$ 的\textbf{数学期望} $E(X)$ 定义为:
		\[
		E(X) = \int_{-\infty}^{+\infty} x f_X(x) \, dx.
		\]
	\end{tcolorbox}
	
	\begin{tcolorbox}[colback=definition!5!white, colframe=definition!75!black, title=方差]
		设 $X$ 的数学期望为 $E(X) = \mu$,则 $X$ 的\textbf{方差} $\text{Var}(X)$ 定义为:
		\[
		\text{Var}(X) = E\left[(X - \mu)^2\right] = \int_{-\infty}^{+\infty} (x - \mu)^2 f_X(x) \, dx.
		\]
	\end{tcolorbox}
	
	\subsection{常用连续型随机变量及其分布}
	
	\begin{tcolorbox}[colback=definition!5!white, colframe=definition!75!black, title=均匀分布]
		若随机变量 $X$ 在区间 $[a, b]$ 上均匀分布,则称 $X$ 服从\textbf{均匀分布},记为 $X \sim U(a, b)$,其概率密度函数为:
		\[
		f_X(x) = 
		\begin{cases}
			\frac{1}{b - a}, & a \leq x \leq b, \\
			0, & \text{otherwise}.
		\end{cases}
		\]
	\end{tcolorbox}
	
	\begin{tcolorbox}[colback=definition!5!white, colframe=definition!75!black, title=正态分布]
		若随机变量 $X$ 的概率密度函数为
		\[
		f_X(x) = \frac{1}{\sqrt{2\pi \sigma^2}} \exp\left(-\frac{(x - \mu)^2}{2\sigma^2}\right),
		\]
		则称 $X$ 服从\textbf{正态分布},记为 $X \sim N(\mu, \sigma^2)$。其中,$\mu$ 为均值,$\sigma^2$ 为方差。
	\end{tcolorbox}
	
	\begin{tcolorbox}[colback=definition!5!white, colframe=definition!75!black, title=指数分布]
		若随机变量 $X$ 的概率密度函数为
		\[
		f_X(x) = 
		\begin{cases}
			\lambda e^{-\lambda x}, & x \geq 0, \\
			0, & x < 0,
		\end{cases}
		\]
		则称 $X$ 服从参数为 $\lambda$ 的\textbf{指数分布},记为 $X \sim \text{Exp}(\lambda)$。
	\end{tcolorbox}
	
	\begin{tcolorbox}[colback=definition!5!white, colframe=definition!75!black, title=对数正态分布]
		若随机变量 $Y = \ln X$ 服从正态分布 $N(\mu, \sigma^2)$,则称 $X$ 服从\textbf{对数正态分布},记为 $X \sim \text{Lognormal}(\mu, \sigma^2)$,其概率密度函数为:
		\[
		f_X(x) = 
		\begin{cases}
			\frac{1}{x \sqrt{2\pi \sigma^2}} \exp\left(-\frac{(\ln x - \mu)^2}{2\sigma^2}\right), & x > 0, \\
			0, & x \leq 0.
		\end{cases}
		\]
	\end{tcolorbox}
	
	\subsection{连续型随机变量函数的分布}
	
	\begin{tcolorbox}[colback=definition!5!white, colframe=definition!75!black, title=连续型随机变量函数的分布]
		设 $X$ 是连续型随机变量,其概率密度函数为 $f_X(x)$,且定义 $Y = g(X)$。若 $g$ 是单调可导函数,则 $Y$ 的概率密度函数为:
		\[
		f_Y(y) = f_X(g^{-1}(y)) \left| \frac{d}{dy}g^{-1}(y) \right|.
		\]
		该公式用于求连续型随机变量的函数分布。
	\end{tcolorbox}
	
	\subsection{其他数字特征}
	
	\begin{tcolorbox}[colback=definition!5!white, colframe=definition!75!black, title=矩]
		设 $X$ 是一个随机变量,若 $E(X^k)$ 存在,则称 $E(X^k)$ 为 $X$ 的 $k$ 阶矩。
	\end{tcolorbox}
	
	\begin{tcolorbox}[colback=definition!5!white, colframe=definition!75!black, title=中心矩]
		设 $X$ 的数学期望为 $\mu$,若 $E((X - \mu)^k)$ 存在,则称 $E((X - \mu)^k)$ 为 $X$ 的 $k$ 阶中心矩。
	\end{tcolorbox}
	
	\begin{tcolorbox}[colback=definition!5!white, colframe=definition!75!black, title=偏度与峰度]
		\begin{itemize}
			\item \textbf{偏度}(Skewness)描述分布的对称性,偏度为 $0$ 表示分布关于均值对称,偏度大于 $0$ 表示分布右偏,偏度小于 $0$ 表示分布左偏。
			\[
			\text{Skewness} = \frac{E[(X - \mu)^3]}{\sigma^3}.
			\]
			\item \textbf{峰度}(Kurtosis)描述分布的陡峭程度,正态分布的峰度为 $3$,峰度大于 $3$ 表示分布更集中,峰度小于 $3$ 表示分布更平坦。
			\[
			\text{Kurtosis} = \frac{E[(X - \mu)^4]}{\sigma^4}.
			\]
		\end{itemize}
	\end{tcolorbox}
	
	% 第四章:随机向量
	
	\section{随机向量}
	
	\subsection{二维随机变量及其联合分布}
	
	\begin{tcolorbox}[colback=definition!5!white, colframe=definition!75!black, title=二维离散型随机变量及联合分布函数]
		设 $(X, Y)$ 是二维离散型随机变量,其联合分布函数定义为:
		\[
		F_{X,Y}(x, y) = P(X \leq x, Y \leq y).
		\]
		联合分布函数 $F_{X,Y}(x, y)$ 满足:
		\begin{enumerate}
			\item $\lim_{x \to -\infty, y \to -\infty} F_{X,Y}(x, y) = 0$;
			\item $\lim_{x \to +\infty, y \to +\infty} F_{X,Y}(x, y) = 1$;
			\item $F_{X,Y}(x, y)$ 是非减的。
		\end{enumerate}
	\end{tcolorbox}
	
	\begin{tcolorbox}[colback=definition!5!white, colframe=definition!75!black, title=二维离散型随机变量的联合分布律]
		设 $(X, Y)$ 是二维离散型随机变量,取值为 $(x_i, y_j)$。则其联合分布律为:
		\[
		P(X = x_i, Y = y_j) = p(x_i, y_j),
		\]
		满足
		\[
		\sum_{i} \sum_{j} p(x_i, y_j) = 1.
		\]
	\end{tcolorbox}
	
	\begin{tcolorbox}[colback=definition!5!white, colframe=definition!75!black, title=二维连续型随机变量及其联合密度函数]
		若随机变量 $(X, Y)$ 的联合分布函数 $F_{X,Y}(x, y)$ 处处可导,则存在联合密度函数 $f_{X,Y}(x, y)$ 使得:
		\[
		f_{X,Y}(x, y) = \frac{\partial^2 F_{X,Y}(x, y)}{\partial x \, \partial y},
		\]
		且满足
		\[
		\iint_{-\infty}^{+\infty} f_{X,Y}(x, y) \, dx \, dy = 1.
		\]
	\end{tcolorbox}
	
	\subsection{边缘分布、随机变量的独立性和条件分布}
	
	\begin{tcolorbox}[colback=definition!5!white, colframe=definition!75!black, title=边缘分布函数]
		设 $(X, Y)$ 是二维随机变量,若 $F_{X,Y}(x, y)$ 为其联合分布函数,则 $X$ 和 $Y$ 的边缘分布函数分别定义为:
		\[
		F_X(x) = \lim_{y \to +\infty} F_{X,Y}(x, y), \quad F_Y(y) = \lim_{x \to +\infty} F_{X,Y}(x, y).
		\]
	\end{tcolorbox}
	
	\begin{tcolorbox}[colback=definition!5!white, colframe=definition!75!black, title=边缘分布律和边缘密度函数]
		对于二维离散型随机变量 $(X, Y)$,$X$ 的边缘分布律为:
		\[
		P(X = x_i) = \sum_{j} P(X = x_i, Y = y_j).
		\]
		对于二维连续型随机变量 $(X, Y)$,$X$ 的边缘密度函数为:
		\[
		f_X(x) = \int_{-\infty}^{+\infty} f_{X,Y}(x, y) \, dy.
		\]
	\end{tcolorbox}
	
	\begin{tcolorbox}[colback=definition!5!white, colframe=definition!75!black, title=随机变量的相互独立性]
		如果对于任意的 $x$ 和 $y$,有 $P(X \leq x, Y \leq y) = P(X \leq x) \cdot P(Y \leq y)$,则称 $X$ 和 $Y$ 相互独立。对于连续型随机变量,若 $f_{X,Y}(x, y) = f_X(x) \cdot f_Y(y)$,则 $X$ 和 $Y$ 相互独立。
	\end{tcolorbox}
	
	\subsection{二维随机变量函数的分布}
	
	\begin{tcolorbox}[colback=definition!5!white, colframe=definition!75!black, title=二维离散型随机变量函数的分布]
		设 $(X, Y)$ 是二维离散型随机变量,$Z = g(X, Y)$ 是其函数,则 $Z$ 的概率分布可以通过 $X$ 和 $Y$ 的联合分布律求出。
	\end{tcolorbox}
	
	\begin{tcolorbox}[colback=definition!5!white, colframe=definition!75!black, title=二维连续型随机变量函数的分布]
		设 $(X, Y)$ 是二维连续型随机变量,$Z = g(X, Y)$ 是其函数。若 $g$ 可逆且可导,则 $Z$ 的概率密度可以通过变量变换公式得到。
	\end{tcolorbox}
	
	\subsection{随机向量的数字特征}
	
	\begin{tcolorbox}[colback=definition!5!white, colframe=definition!75!black, title=二维随机变量函数的数学期望]
		设 $(X, Y)$ 是二维随机变量,若 $g(X, Y)$ 是一个函数,则其数学期望定义为:
		\[
		E[g(X, Y)] = 
		\begin{cases}
			\sum_{i} \sum_{j} g(x_i, y_j) p(x_i, y_j), & \text{离散型随机变量}, \\
			\iint_{-\infty}^{+\infty} g(x, y) f_{X,Y}(x, y) \, dx \, dy, & \text{连续型随机变量}.
		\end{cases}
		\]
	\end{tcolorbox}
	
	\begin{tcolorbox}[colback=definition!5!white, colframe=definition!75!black, title=协方差及相关系数]
		设 $X$ 和 $Y$ 的数学期望分别为 $E(X) = \mu_X$ 和 $E(Y) = \mu_Y$,则 $X$ 和 $Y$ 的\textbf{协方差} $\text{Cov}(X, Y)$ 定义为:
		\[
		\text{Cov}(X, Y) = E[(X - \mu_X)(Y - \mu_Y)].
		\]
		$X$ 和 $Y$ 的\textbf{相关系数} $\rho_{X,Y}$ 定义为:
		\[
		\rho_{X,Y} = \frac{\text{Cov}(X, Y)}{\sqrt{\text{Var}(X)} \sqrt{\text{Var}(Y)}}.
		\]
	\end{tcolorbox}
	
	\begin{tcolorbox}[colback=definition!5!white, colframe=definition!75!black, title=期望向量和协方差矩阵]
		若 $(X_1, X_2, \dots, X_n)$ 为 $n$ 维随机向量,则其\textbf{期望向量} $\mu$ 定义为:
		\[
		\mu = \begin{bmatrix} E(X_1) \\ E(X_2) \\ \vdots \\ E(X_n) \end{bmatrix}.
		\]
		其\textbf{协方差矩阵} $\Sigma$ 定义为:
		\[
		\Sigma = \begin{bmatrix} 
			\text{Var}(X_1) & \text{Cov}(X_1, X_2) & \cdots & \text{Cov}(X_1, X_n) \\ 
			\text{Cov}(X_2, X_1) & \text{Var}(X_2) & \cdots & \text{Cov}(X_2, X_n) \\ 
			\vdots & \vdots & \ddots & \vdots \\ 
			\text{Cov}(X_n, X_1) & \text{Cov}(X_n, X_2) & \cdots & \text{Var}(X_n) 
		\end{bmatrix}.
		\]
	\end{tcolorbox}
	
	\begin{tcolorbox}[colback=definition!5!white, colframe=definition!75!black, title=多维正态分布]
		若随机向量 $X = (X_1, X_2, \dots, X_n)^T$ 服从多维正态分布,记为 $X \sim N(\mu, \Sigma)$,其中 $\mu$ 为期望向量,$\Sigma$ 为协方差矩阵。则其概率密度函数为:
		\[
		f_X(x) = \frac{1}{\sqrt{(2\pi)^n |\Sigma|}} \exp\left(-\frac{1}{2} (x - \mu)^T \Sigma^{-1} (x - \mu)\right).
		\]
	\end{tcolorbox}
	
\end{document}